\documentclass[11pt,a4paper]{article}

% -------------------------------------------------
% PACKAGES
% -------------------------------------------------
\usepackage[utf8]{inputenc}
\usepackage[T1]{fontenc}
\usepackage[french]{babel}
\usepackage{amsmath,amssymb}
\usepackage{graphicx}
\usepackage{geometry}
\geometry{margin=2.5cm}

\title{Problème à deux corps gravitationnel :\\
résolution analytique de l'orbite circulaire}
\author{TIPE --- Méthodes d'intégration numérique}
\date{}

\begin{document}
\maketitle

\section{Introduction}

On considère deux corps de masses non négligeables $m_1$ et $m_2$
soumis uniquement à leur interaction gravitationnelle newtonienne.
Le but de cette partie est d'obtenir une \emph{solution analytique}
du mouvement lorsque l'orbite relative est circulaire.
Cette solution servira de référence pour évaluer la précision de différentes
méthodes d'intégration numérique (Euler, Verlet, Runge--Kutta, etc.).

\section{Équations de Newton pour deux corps}

On note $\vec r_1(t)$ et $\vec r_2(t)$ les positions des deux masses
dans un référentiel galiléen, et
\[
\vec r = \vec r_1 - \vec r_2, \qquad r = \|\vec r\|.
\]

La force gravitationnelle exercée par $2$ sur $1$ est
\begin{equation}
\vec F_{1} = G\,\frac{m_1 m_2}{r^3}\,(\vec r_2 - \vec r_1),
\end{equation}
et par la troisième loi de Newton,
\begin{equation}
\vec F_{2} = -\vec F_{1}.
\end{equation}

Les équations de Newton s'écrivent alors
\begin{equation}
\label{eq:newton1}
m_1 \ddot{\vec r}_1 = G\,\frac{m_1 m_2}{r^3}(\vec r_2 - \vec r_1),
\end{equation}
\begin{equation}
\label{eq:newton2}
m_2 \ddot{\vec r}_2 = -G\,\frac{m_1 m_2}{r^3}(\vec r_2 - \vec r_1).
\end{equation}

Ces équations couplées peuvent être simplifiées grâce à une
réduction barycentrique.

\section{Réduction barycentrique}

On introduit la position du barycentre
\begin{equation}
\vec R = \frac{m_1 \vec r_1 + m_2 \vec r_2}{M},
\qquad
M = m_1 + m_2,
\end{equation}
et le vecteur position relatif
\begin{equation}
\vec r = \vec r_1 - \vec r_2.
\end{equation}

On peut inverser ces relations pour exprimer les positions individuelles
en fonction de $\vec R$ et $\vec r$ :
\begin{equation}
\label{eq:r1Rr}
\vec r_1 = \vec R + \frac{m_2}{M}\,\vec r,
\end{equation}
\begin{equation}
\label{eq:r2Rr}
\vec r_2 = \vec R - \frac{m_1}{M}\,\vec r.
\end{equation}

\subsection{Mouvement du barycentre}

En additionnant les équations \eqref{eq:newton1} et \eqref{eq:newton2},
on obtient
\begin{equation}
m_1 \ddot{\vec r}_1 + m_2 \ddot{\vec r}_2 = 0.
\end{equation}
Or, par définition de $\vec R$,
\begin{equation}
M \ddot{\vec R}
= m_1 \ddot{\vec r}_1 + m_2 \ddot{\vec r}_2,
\end{equation}
d'où
\begin{equation}
M \ddot{\vec R} = 0
\quad\Longrightarrow\quad
\ddot{\vec R} = 0.
\end{equation}

Le barycentre décrit donc un mouvement rectiligne uniforme :
\begin{equation}
\label{eq:R(t)}
\vec R(t) = \vec R_0 + \vec V_0\,t,
\end{equation}
où $\vec R_0$ et $\vec V_0$ sont respectivement la position et la
vitesse initiales du barycentre.

\subsection{Mouvement relatif}

On considère maintenant le vecteur relatif
$\vec r = \vec r_1 - \vec r_2$.
En dérivant deux fois par rapport au temps :
\begin{equation}
\ddot{\vec r} = \ddot{\vec r}_1 - \ddot{\vec r}_2.
\end{equation}
En utilisant \eqref{eq:newton1} et \eqref{eq:newton2}, on obtient
\begin{equation}
\ddot{\vec r}_1
= G\,\frac{m_2}{r^3}(\vec r_2 - \vec r_1)
= -G\,\frac{m_2}{r^3}\,\vec r,
\end{equation}
\begin{equation}
\ddot{\vec r}_2
= -G\,\frac{m_1}{r^3}(\vec r_2 - \vec r_1)
= G\,\frac{m_1}{r^3}\,\vec r.
\end{equation}
D'où
\begin{equation}
\ddot{\vec r}
= -G\,\frac{m_2}{r^3}\,\vec r
   -G\,\frac{m_1}{r^3}\,\vec r
= -G\,\frac{(m_1 + m_2)}{r^3}\,\vec r
= -G\,\frac{M}{r^3}\,\vec r.
\end{equation}

On obtient ainsi l'équation du mouvement relatif
\begin{equation}
\label{eq:relative}
\boxed{
\ddot{\vec r} = -G\,\frac{M}{r^3}\,\vec r
}
\end{equation}
qui est l'équation d'une particule soumise à un potentiel central de type képlérien.

\section{Condition d'orbite circulaire}

Nous cherchons des solutions pour lesquelles la distance relative
est constante :
\begin{equation}
r(t) = r_0.
\end{equation}
Le vecteur position relatif s'écrit alors
\begin{equation}
\vec r(t) = r_0\,\hat u_r(t),
\end{equation}
où $\hat u_r(t)$ est un vecteur unitaire décrivant la direction radiale.

Pour un mouvement circulaire uniforme dans un plan, la dérivée seconde
de la position est une accélération centripète :
\begin{equation}
\ddot{\vec r} = -r_0 \omega^2 \hat u_r,
\end{equation}
où $\omega$ est la vitesse angulaire (constante).

En identifiant avec l'équation du mouvement relatif \eqref{eq:relative},
on obtient
\begin{equation}
-r_0 \omega^2 \hat u_r
= -G\,\frac{M}{r_0^2}\,\hat u_r,
\end{equation}
d'où
\begin{equation}
r_0 \omega^2 = \frac{G M}{r_0^2}
\quad\Longrightarrow\quad
\boxed{
\omega = \sqrt{\frac{G M}{r_0^3}}
}.
\end{equation}

\section{Expression explicite de la trajectoire relative}

On introduit dans le plan de l'orbite les vecteurs unitaires
\begin{equation}
\hat u_r = \frac{\vec r_0}{r_0},
\qquad
\hat u_\theta = (-u_{r,y},\,u_{r,x}),
\end{equation}
où $\hat u_\theta$ est obtenu par rotation de $\hat u_r$ d'un angle
$\pi/2$ dans le sens trigonométrique.

La solution du mouvement relatif s'écrit alors
\begin{equation}
\boxed{
\vec r_{\mathrm{rel}}(t)
= r_0\Big[
\cos(\omega t)\,\hat u_r
+ \sin(\omega t)\,\hat u_\theta
\Big]
}.
\end{equation}

Le barycentre suit le mouvement rectiligne uniforme \eqref{eq:R(t)}.
Les positions des deux corps s'en déduisent à partir de
\eqref{eq:r1Rr} et \eqref{eq:r2Rr} :
\begin{equation}
\boxed{
\vec r_1(t)
= \vec R(t) + \frac{m_2}{M}\,\vec r_{\mathrm{rel}}(t)
}
\end{equation}
\begin{equation}
\boxed{
\vec r_2(t)
= \vec R(t) - \frac{m_1}{M}\,\vec r_{\mathrm{rel}}(t)
}
\end{equation}
Ce sont ces expressions qui sont implémentées dans le code numérique
pour obtenir la \emph{solution analytique de référence}.

\section{Vitesses initiales pour une orbite circulaire}

On dérive la position relative :
\begin{equation}
\dot{\vec r}_{\mathrm{rel}}(t)
= r_0 \omega
\Big[
-\sin(\omega t)\,\hat u_r
+ \cos(\omega t)\,\hat u_\theta
\Big].
\end{equation}
En particulier, à l'instant initial $t=0$ :
\begin{equation}
\boxed{
\dot{\vec r}_{\mathrm{rel}}(0)
= r_0 \omega\,\hat u_\theta
}
\end{equation}
La norme de la vitesse relative est donc
\begin{equation}
\boxed{
v_{\mathrm{rel}}
= \|\dot{\vec r}_{\mathrm{rel}}(0)\|
= r_0 \omega
= \sqrt{\frac{G M}{r_0}}
}.
\end{equation}

Les vitesses individuelles se déduisent des relations barycentriques.
En dérivant \eqref{eq:r1Rr} et \eqref{eq:r2Rr}, on obtient
\begin{equation}
\vec v_1 = \dot{\vec R} + \frac{m_2}{M}\dot{\vec r}
= \vec V_0 + \frac{m_2}{M}\,\vec v_{\mathrm{rel}},
\end{equation}
\begin{equation}
\vec v_2 = \dot{\vec R} - \frac{m_1}{M}\dot{\vec r}
= \vec V_0 - \frac{m_1}{M}\,\vec v_{\mathrm{rel}},
\end{equation}
où $\vec v_{\mathrm{rel}} = \dot{\vec r}_{\mathrm{rel}}(0)$.

Si l'on choisit un barycentre au repos ($\vec V_0 = \vec 0$), ce qui
est souvent le cas dans la simulation numérique, les vitesses initiales
des deux corps s'écrivent
\begin{equation}
\boxed{
\vec v_1 = \frac{m_2}{M}\,\vec v_{\mathrm{rel}},
\qquad
\vec v_2 = -\frac{m_1}{M}\,\vec v_{\mathrm{rel}}.
}
\end{equation}

Ces expressions fournissent les conditions initiales exactes pour une
orbite circulaire, utilisées dans le programme afin de comparer les
méthodes d'intégration numérique (Euler, Verlet, RK4) à la solution
analytique.

\end{document}
